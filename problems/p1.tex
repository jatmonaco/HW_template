\section{Problem 1}
% --- Wrap figure environment --- % 
\begin{wrapfigure}{r}{0.4\textwidth}
    \begin{center}
        \includegraphics[width=0.8\linewidth]{example-image}
    \end{center}
    \caption{Wrapfigure environment}
\end{wrapfigure}
% Placeholder text with citations 
\lipsum[26] \cite{fouhey_2010MachineLearning_2018}.
\lipsum[55]
% --- Math environment with cancel and annotation --- % 
\[\begin{WithArrows}
    \langle (S_n - Y)^2 \rangle & = \langle S_n^2 + Y^2 - 2S_n Y \rangle \Arrow{Expectation operator \\ is linear} \\ 
    & = \langle S_n^2 \rangle + \langle Y^2 \rangle - 2 \langle S_n Y \rangle \\ 
    & \xrightarrow[]{n \rightarrow \infty} \underbrace{\langle \left( \sum\limits_{i=1}^{n}a_i X_i  \right)^2 \rangle}_{\text{cross terms are zero}} + \sum\limits_{i=1}^{n} \alpha_i^2 - 2 \langle \sum\limits_{i=1}^{n} \alpha_i X_i Y \rangle \\ 
    & = \langle \sum\limits_{i=1}^{n} \alpha_i^2 X_i^2 \rangle + \sum\limits_{i=1}^{n} \alpha_i^2 - 2 \langle \sum\limits_{i=1}^{n} \alpha_i X_i Y \rangle \Arrow{Only \(X_i\) and \(Y\) \\ are non-deterministic} \\ 
    & = \sum\limits_{i=1}^{n} \alpha_i^2 \underbrace{\cancelto{1}{\langle  X_i^2 \rangle}}_{\var(X_i) = 1} + \sum\limits_{i=1}^{n} \alpha_i^2 - 2 \sum\limits_{i=1}^{\infty} \alpha_i \underbrace{\langle Y X_i \rangle}_{\alpha_i} \\ 
    & = 2\sum\limits_{i=1}^{n} \alpha_i^2 - 2 \sum\limits_{i=1}^{n} \alpha_i^2 \\ 
    & = 0
\end{WithArrows}\]
% 
\subsection{Part A}
% --- Remark environment --- % 
\begin{remark}[test]
    \lipsum[55]
\end{remark}
% --- Align math environment --- % 
\begin{align*}
    f_Y(y) = 
     \begin{cases}
         \frac{1}{\sqrt{\pi \left( \sqrt{2}y+1 \right)}}\e^{-\frac{1}{2}\left( \sqrt{2}y+1 \right)} &  -\frac{1}{\sqrt{2}} \leq y < \infty \\ 
        0     & \mathrm{otherwise}
    \end{cases}
\end{align*}
%
\lipsum[23]
% --- Definition environment --- %
\begin{definition}[test]
    \lipsum[44]
\end{definition}
%
\lipsum[32]
%
\subsection{Part B}
% --- Figure environment --- %
\begin{figure}[h!tbp]
    \centering
    \begin{subfigure}[b]{0.45\linewidth}
        \centering
        \includegraphics[width=\linewidth]{example-image-a}
        \subcaption{Example A}
    \end{subfigure}
    \begin{subfigure}[b]{0.45\linewidth}
        \centering
        \includegraphics[width=\linewidth]{example-image-a}
        \subcaption{Example B}
    \end{subfigure}
    \caption{Figure environment \cite{galvez_MachinelearningDataSet_2019}.}
    \label{fig:p2_a}
\end{figure}
% --- Theorem  environment --- %
\begin{theorem}[test]
    \lipsum[2]
\end{theorem}
%
\lipsum[34]
% --- Example environment --- %
\begin{eg}[test]
    \lipsum[36]
\end{eg}